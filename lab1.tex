% Options for packages loaded elsewhere
\PassOptionsToPackage{unicode}{hyperref}
\PassOptionsToPackage{hyphens}{url}
%
\documentclass[
]{article}
\usepackage{lmodern}
\usepackage{amssymb,amsmath}
\usepackage{ifxetex,ifluatex}
\ifnum 0\ifxetex 1\fi\ifluatex 1\fi=0 % if pdftex
  \usepackage[T1]{fontenc}
  \usepackage[utf8]{inputenc}
  \usepackage{textcomp} % provide euro and other symbols
\else % if luatex or xetex
  \usepackage{unicode-math}
  \defaultfontfeatures{Scale=MatchLowercase}
  \defaultfontfeatures[\rmfamily]{Ligatures=TeX,Scale=1}
\fi
% Use upquote if available, for straight quotes in verbatim environments
\IfFileExists{upquote.sty}{\usepackage{upquote}}{}
\IfFileExists{microtype.sty}{% use microtype if available
  \usepackage[]{microtype}
  \UseMicrotypeSet[protrusion]{basicmath} % disable protrusion for tt fonts
}{}
\makeatletter
\@ifundefined{KOMAClassName}{% if non-KOMA class
  \IfFileExists{parskip.sty}{%
    \usepackage{parskip}
  }{% else
    \setlength{\parindent}{0pt}
    \setlength{\parskip}{6pt plus 2pt minus 1pt}}
}{% if KOMA class
  \KOMAoptions{parskip=half}}
\makeatother
\usepackage{xcolor}
\IfFileExists{xurl.sty}{\usepackage{xurl}}{} % add URL line breaks if available
\IfFileExists{bookmark.sty}{\usepackage{bookmark}}{\usepackage{hyperref}}
\hypersetup{
  pdftitle={Lab 1},
  pdfauthor={Carmen Canedo},
  hidelinks,
  pdfcreator={LaTeX via pandoc}}
\urlstyle{same} % disable monospaced font for URLs
\usepackage[margin=1in]{geometry}
\usepackage{color}
\usepackage{fancyvrb}
\newcommand{\VerbBar}{|}
\newcommand{\VERB}{\Verb[commandchars=\\\{\}]}
\DefineVerbatimEnvironment{Highlighting}{Verbatim}{commandchars=\\\{\}}
% Add ',fontsize=\small' for more characters per line
\usepackage{framed}
\definecolor{shadecolor}{RGB}{248,248,248}
\newenvironment{Shaded}{\begin{snugshade}}{\end{snugshade}}
\newcommand{\AlertTok}[1]{\textcolor[rgb]{0.94,0.16,0.16}{#1}}
\newcommand{\AnnotationTok}[1]{\textcolor[rgb]{0.56,0.35,0.01}{\textbf{\textit{#1}}}}
\newcommand{\AttributeTok}[1]{\textcolor[rgb]{0.77,0.63,0.00}{#1}}
\newcommand{\BaseNTok}[1]{\textcolor[rgb]{0.00,0.00,0.81}{#1}}
\newcommand{\BuiltInTok}[1]{#1}
\newcommand{\CharTok}[1]{\textcolor[rgb]{0.31,0.60,0.02}{#1}}
\newcommand{\CommentTok}[1]{\textcolor[rgb]{0.56,0.35,0.01}{\textit{#1}}}
\newcommand{\CommentVarTok}[1]{\textcolor[rgb]{0.56,0.35,0.01}{\textbf{\textit{#1}}}}
\newcommand{\ConstantTok}[1]{\textcolor[rgb]{0.00,0.00,0.00}{#1}}
\newcommand{\ControlFlowTok}[1]{\textcolor[rgb]{0.13,0.29,0.53}{\textbf{#1}}}
\newcommand{\DataTypeTok}[1]{\textcolor[rgb]{0.13,0.29,0.53}{#1}}
\newcommand{\DecValTok}[1]{\textcolor[rgb]{0.00,0.00,0.81}{#1}}
\newcommand{\DocumentationTok}[1]{\textcolor[rgb]{0.56,0.35,0.01}{\textbf{\textit{#1}}}}
\newcommand{\ErrorTok}[1]{\textcolor[rgb]{0.64,0.00,0.00}{\textbf{#1}}}
\newcommand{\ExtensionTok}[1]{#1}
\newcommand{\FloatTok}[1]{\textcolor[rgb]{0.00,0.00,0.81}{#1}}
\newcommand{\FunctionTok}[1]{\textcolor[rgb]{0.00,0.00,0.00}{#1}}
\newcommand{\ImportTok}[1]{#1}
\newcommand{\InformationTok}[1]{\textcolor[rgb]{0.56,0.35,0.01}{\textbf{\textit{#1}}}}
\newcommand{\KeywordTok}[1]{\textcolor[rgb]{0.13,0.29,0.53}{\textbf{#1}}}
\newcommand{\NormalTok}[1]{#1}
\newcommand{\OperatorTok}[1]{\textcolor[rgb]{0.81,0.36,0.00}{\textbf{#1}}}
\newcommand{\OtherTok}[1]{\textcolor[rgb]{0.56,0.35,0.01}{#1}}
\newcommand{\PreprocessorTok}[1]{\textcolor[rgb]{0.56,0.35,0.01}{\textit{#1}}}
\newcommand{\RegionMarkerTok}[1]{#1}
\newcommand{\SpecialCharTok}[1]{\textcolor[rgb]{0.00,0.00,0.00}{#1}}
\newcommand{\SpecialStringTok}[1]{\textcolor[rgb]{0.31,0.60,0.02}{#1}}
\newcommand{\StringTok}[1]{\textcolor[rgb]{0.31,0.60,0.02}{#1}}
\newcommand{\VariableTok}[1]{\textcolor[rgb]{0.00,0.00,0.00}{#1}}
\newcommand{\VerbatimStringTok}[1]{\textcolor[rgb]{0.31,0.60,0.02}{#1}}
\newcommand{\WarningTok}[1]{\textcolor[rgb]{0.56,0.35,0.01}{\textbf{\textit{#1}}}}
\usepackage{graphicx,grffile}
\makeatletter
\def\maxwidth{\ifdim\Gin@nat@width>\linewidth\linewidth\else\Gin@nat@width\fi}
\def\maxheight{\ifdim\Gin@nat@height>\textheight\textheight\else\Gin@nat@height\fi}
\makeatother
% Scale images if necessary, so that they will not overflow the page
% margins by default, and it is still possible to overwrite the defaults
% using explicit options in \includegraphics[width, height, ...]{}
\setkeys{Gin}{width=\maxwidth,height=\maxheight,keepaspectratio}
% Set default figure placement to htbp
\makeatletter
\def\fps@figure{htbp}
\makeatother
\setlength{\emergencystretch}{3em} % prevent overfull lines
\providecommand{\tightlist}{%
  \setlength{\itemsep}{0pt}\setlength{\parskip}{0pt}}
\setcounter{secnumdepth}{-\maxdimen} % remove section numbering

\title{Lab 1}
\author{Carmen Canedo}
\date{May 23, 2020}

\begin{document}
\maketitle

\hypertarget{about}{%
\section{About}\label{about}}

For this lab, I used the GPS tracking app on my iPhone, myTracker, to
trace myself walking a straight path outside of my house in Nashville,
TN. I tried to maintain a steady pace as I walked up and down the
street.

Below you will find an analysis of my path that was created using R
statistical software. My code is available below, but my step by step
process can be found on
\href{https://github.com/carmen-canedo/stat-202-lab1}{my GitHub page}.

\hypertarget{necessary-packages}{%
\section{Necessary Packages}\label{necessary-packages}}

\begin{Shaded}
\begin{Highlighting}[]
\KeywordTok{library}\NormalTok{(mdsr)}
\KeywordTok{library}\NormalTok{(XML)}
\KeywordTok{library}\NormalTok{(OpenStreetMap)}
\KeywordTok{library}\NormalTok{(lubridate)}
\KeywordTok{library}\NormalTok{(ggmap)}
\KeywordTok{library}\NormalTok{(raster)}
\KeywordTok{library}\NormalTok{(sp)}
\end{Highlighting}
\end{Shaded}

\hypertarget{getting-the-data-ready}{%
\section{Getting the data ready}\label{getting-the-data-ready}}

\hypertarget{loading-in-data}{%
\subsection{Loading in data}\label{loading-in-data}}

\begin{Shaded}
\begin{Highlighting}[]
\NormalTok{walking <-}\StringTok{ }\KeywordTok{read.csv}\NormalTok{(}\StringTok{"lab-1.csv"}\NormalTok{, }\DataTypeTok{header =} \OtherTok{TRUE}\NormalTok{)}
\end{Highlighting}
\end{Shaded}

\hypertarget{cleaning-data}{%
\subsection{Cleaning data}\label{cleaning-data}}

\#\#\#Getting rid of unnecessary columns

\begin{Shaded}
\begin{Highlighting}[]
\NormalTok{walking <-}\StringTok{ }\NormalTok{walking }\OperatorTok\StringTok{ }
\StringTok{  }\NormalTok{dplyr}\OperatorTok{::}\KeywordTok{select}\NormalTok{(}\OperatorTok{-}\NormalTok{type, }\OperatorTok{-}\NormalTok{desc, }\OperatorTok{-}\NormalTok{name)}
\end{Highlighting}
\end{Shaded}

\hypertarget{making-column-names-simpler}{%
\subsubsection{Making column names
simpler}\label{making-column-names-simpler}}

\begin{Shaded}
\begin{Highlighting}[]
\NormalTok{walking <-}\StringTok{ }\NormalTok{walking }\OperatorTok\StringTok{ }
\StringTok{  }\KeywordTok{rename}\NormalTok{(}\DataTypeTok{altitude =}\NormalTok{ altitude..ft.) }\OperatorTok\StringTok{ }
\StringTok{  }\KeywordTok{rename}\NormalTok{(}\DataTypeTok{speed =}\NormalTok{ speed..mph.) }\OperatorTok\StringTok{ }
\StringTok{  }\KeywordTok{rename}\NormalTok{(}\DataTypeTok{distance_mi =}\NormalTok{ distance..mi.) }\OperatorTok\StringTok{ }
\StringTok{  }\KeywordTok{rename}\NormalTok{(}\DataTypeTok{distance_int_ft =}\NormalTok{ distance_interval..ft.)}
\end{Highlighting}
\end{Shaded}

\hypertarget{summary-stats-calculations}{%
\section{Summary stats calculations}\label{summary-stats-calculations}}

\begin{Shaded}
\begin{Highlighting}[]
\NormalTok{sum_latitude <-}\StringTok{ }\KeywordTok{favstats}\NormalTok{( }\OperatorTok{~}\StringTok{ }\NormalTok{latitude, }\DataTypeTok{data =}\NormalTok{ walking)}

\NormalTok{sum_longitude <-}\StringTok{ }\KeywordTok{favstats}\NormalTok{( }\OperatorTok{~}\StringTok{ }\NormalTok{longitude, }\DataTypeTok{data =}\NormalTok{ walking)}

\NormalTok{sum_altitude <-}\StringTok{ }\KeywordTok{favstats}\NormalTok{( }\OperatorTok{~}\StringTok{ }\NormalTok{altitude, }\DataTypeTok{data =}\NormalTok{ walking)}

\NormalTok{sum_speed <-}\StringTok{ }\KeywordTok{favstats}\NormalTok{( }\OperatorTok{~}\StringTok{ }\NormalTok{speed, }\DataTypeTok{data =}\NormalTok{ walking)}

\NormalTok{sum_distance_mi <-}\StringTok{ }\KeywordTok{favstats}\NormalTok{( }\OperatorTok{~}\StringTok{ }\NormalTok{distance_mi, }\DataTypeTok{data =}\NormalTok{ walking)}

\NormalTok{sum_dist_int_ft <-}\StringTok{ }\KeywordTok{favstats}\NormalTok{( }\OperatorTok{~}\StringTok{ }\NormalTok{distance_int_ft, }\DataTypeTok{data =}\NormalTok{ walking)}
\end{Highlighting}
\end{Shaded}

\hypertarget{results}{%
\subsection{Results}\label{results}}

\begin{Shaded}
\begin{Highlighting}[]
\CommentTok{# Results}
\NormalTok{sum_latitude}
\end{Highlighting}
\end{Shaded}

\begin{verbatim}
##       min       Q1  median       Q3      max    mean           sd   n missing
##  36.18001 36.18029 36.1806 36.18091 36.18117 36.1806 0.0003514294 221       0
\end{verbatim}

\begin{Shaded}
\begin{Highlighting}[]
\NormalTok{sum_longitude}
\end{Highlighting}
\end{Shaded}

\begin{verbatim}
##        min        Q1   median        Q3       max     mean           sd   n
##  -86.74297 -86.74294 -86.7429 -86.74286 -86.74278 -86.7429 4.645446e-05 221
##  missing
##        0
\end{verbatim}

\begin{Shaded}
\begin{Highlighting}[]
\NormalTok{sum_altitude}
\end{Highlighting}
\end{Shaded}

\begin{verbatim}
##    min    Q1 median    Q3   max     mean       sd   n missing
##  500.8 505.8  511.2 512.4 516.1 509.3181 4.041696 221       0
\end{verbatim}

\begin{Shaded}
\begin{Highlighting}[]
\NormalTok{sum_speed}
\end{Highlighting}
\end{Shaded}

\begin{verbatim}
##  min    Q1 median    Q3 max     mean       sd   n missing
##    0 2.275    2.7 3.325  10 2.839545 1.226507 220       1
\end{verbatim}

\begin{Shaded}
\begin{Highlighting}[]
\NormalTok{sum_distance_mi}
\end{Highlighting}
\end{Shaded}

\begin{verbatim}
##  min    Q1 median    Q3   max       mean         sd   n missing
##    0 0.047   0.09 0.137 0.181 0.09093213 0.05260591 221       0
\end{verbatim}

\begin{Shaded}
\begin{Highlighting}[]
\NormalTok{sum_dist_int_ft}
\end{Highlighting}
\end{Shaded}

\begin{verbatim}
##  min   Q1 median   Q3   max    mean       sd   n missing
##    0 3.14   4.01 5.29 14.58 4.32362 1.937173 221       0
\end{verbatim}

\hypertarget{analysis}{%
\subsubsection{Analysis}\label{analysis}}

\begin{itemize}
\tightlist
\item
  Question 1:

  \begin{itemize}
  \tightlist
  \item
    The standard deviation is larger for latitude.
  \end{itemize}
\item
  Question 2:

  \begin{itemize}
  \tightlist
  \item
    This tells us that the latitude moves farther from the mean
    latitude.
  \end{itemize}
\end{itemize}

\hypertarget{creating-latitude-v.-longitude-scatter-plot}{%
\section{Creating Latitude v. Longitude Scatter
Plot}\label{creating-latitude-v.-longitude-scatter-plot}}

\begin{Shaded}
\begin{Highlighting}[]
\NormalTok{lat_v_long <-}\StringTok{ }\NormalTok{walking }\OperatorTok
\StringTok{  }\KeywordTok{ggplot}\NormalTok{(}\KeywordTok{aes}\NormalTok{(}\DataTypeTok{x =}\NormalTok{ longitude, }\DataTypeTok{y =}\NormalTok{ latitude)) }\OperatorTok{+}
\StringTok{  }\KeywordTok{geom_point}\NormalTok{(}\DataTypeTok{alpha =} \FloatTok{0.8}\NormalTok{, }\KeywordTok{aes}\NormalTok{(}\DataTypeTok{color =}\NormalTok{ speed), }\DataTypeTok{size =} \DecValTok{3}\NormalTok{) }\OperatorTok{+}
\StringTok{  }\KeywordTok{scale_color_gradient}\NormalTok{(}\DataTypeTok{low =} \StringTok{"blue"}\NormalTok{, }\DataTypeTok{high =} \StringTok{"red"}\NormalTok{) }\OperatorTok{+}
\StringTok{  }\KeywordTok{theme_minimal}\NormalTok{() }\OperatorTok{+}
\StringTok{  }\KeywordTok{labs}\NormalTok{(}\DataTypeTok{title =} \StringTok{"Longitude versus Latitude"}\NormalTok{,}
       \DataTypeTok{subtitle =} \StringTok{"Carmen Canedo"}\NormalTok{,}
       \DataTypeTok{caption =} \StringTok{"STAT 202, Summer 2020"}\NormalTok{,}
       \DataTypeTok{x =} \StringTok{"Longitude"}\NormalTok{,}
       \DataTypeTok{y =} \StringTok{"Latitude"}\NormalTok{,}
       \DataTypeTok{color =} \StringTok{"Speed (mph)"}\NormalTok{)}

\NormalTok{lat_v_long}
\end{Highlighting}
\end{Shaded}

\includegraphics{lab1_files/figure-latex/unnamed-chunk-7-1.pdf}

\hypertarget{adding-line-of-best-fit}{%
\subsection{Adding Line of Best Fit}\label{adding-line-of-best-fit}}

\begin{Shaded}
\begin{Highlighting}[]
\NormalTok{lat_v_long <-}\StringTok{ }\NormalTok{lat_v_long }\OperatorTok{+}
\StringTok{  }\KeywordTok{geom_smooth}\NormalTok{(}\DataTypeTok{method =} \StringTok{"lm"}\NormalTok{)}

\NormalTok{lat_v_long}
\end{Highlighting}
\end{Shaded}

\begin{verbatim}
## `geom_smooth()` using formula 'y ~ x'
\end{verbatim}

\includegraphics{lab1_files/figure-latex/unnamed-chunk-8-1.pdf}

\hypertarget{simple-linear-regression-results}{%
\subsection{Simple Linear Regression
Results}\label{simple-linear-regression-results}}

\begin{Shaded}
\begin{Highlighting}[]
\CommentTok{# Calculating model}
\NormalTok{model <-}\StringTok{ }\KeywordTok{lm}\NormalTok{(latitude }\OperatorTok{~}\StringTok{ }\NormalTok{longitude, }\DataTypeTok{data =}\NormalTok{ walking)}

\CommentTok{# Finding correlation coefficient}
\KeywordTok{coef}\NormalTok{(model)}
\end{Highlighting}
\end{Shaded}

\begin{verbatim}
## (Intercept)   longitude 
##  651.652871    7.095362
\end{verbatim}

\emph{Formula for line of best fit:} \n
\(latitude = 651.653 + 7.0954(longitude)\)

\hypertarget{analysis-1}{%
\subsubsection{Analysis}\label{analysis-1}}

\begin{itemize}
\tightlist
\item
  Is the line of best fit a good tool to estimate the path traveled? Why
  or why not?
\item
  How does the correlation help you answer part b?
\end{itemize}

\hypertarget{mapping-the-route}{%
\section{Mapping the route}\label{mapping-the-route}}

I referenced exercises from \href{https://rpubs.com/ials2un/gpx1}{here}

\hypertarget{getting-the-data}{%
\subsection{Getting the data}\label{getting-the-data}}

In order to ensure that all the values work when mapped, this equation
places the vectors correctly.

\begin{Shaded}
\begin{Highlighting}[]
\CommentTok{# Function to shift vectors}
\NormalTok{shift_vec <-}\StringTok{ }\ControlFlowTok{function}\NormalTok{(vector, shift) \{}
  \ControlFlowTok{if}\NormalTok{ (}\KeywordTok{length}\NormalTok{(vec) }\OperatorTok{<=}\StringTok{ }\KeywordTok{abs}\NormalTok{(shift)) \{}
    \KeywordTok{rep}\NormalTok{(}\OtherTok{NA}\NormalTok{, }\KeywordTok{length}\NormalTok{(vec))}
\NormalTok{  \} }\ControlFlowTok{else}\NormalTok{ \{}
    \ControlFlowTok{if}\NormalTok{ (shift }\OperatorTok{>=}\StringTok{ }\DecValTok{0}\NormalTok{) \{}
      \KeywordTok{c}\NormalTok{(}\KeywordTok{rep}\NormalTok{(}\OtherTok{NA}\NormalTok{, shift), vec[}\DecValTok{1}\OperatorTok{:}\NormalTok{(}\KeywordTok{length}\NormalTok{(vec) }\OperatorTok{-}\StringTok{ }\NormalTok{shift)])}
\NormalTok{    \} }\ControlFlowTok{else}\NormalTok{ \{}
      \KeywordTok{c}\NormalTok{(vec[(}\KeywordTok{abs}\NormalTok{(shift) }\OperatorTok{+}\StringTok{ }\DecValTok{1}\NormalTok{)}\OperatorTok{:}\KeywordTok{length}\NormalTok{(vec)])}
\NormalTok{    \}}
\NormalTok{  \}}
\NormalTok{\}}
\end{Highlighting}
\end{Shaded}

\hypertarget{reading-in-gpx-file}{%
\subsection{Reading in GPX file}\label{reading-in-gpx-file}}

\begin{Shaded}
\begin{Highlighting}[]
\CommentTok{# Limits to 10 digits}
\KeywordTok{options}\NormalTok{(}\DataTypeTok{digits =} \DecValTok{10}\NormalTok{)}

\CommentTok{# Parsing the GPX file}
\NormalTok{parsed_file <-}\StringTok{ }\KeywordTok{htmlTreeParse}\NormalTok{(}\DataTypeTok{file =} \StringTok{"lab-1-raw-data.gpx"}\NormalTok{,}
                             \DataTypeTok{error =} \ControlFlowTok{function}\NormalTok{(...) \{\},}
                             \DataTypeTok{useInternalNodes =} \OtherTok{TRUE}\NormalTok{)}

\CommentTok{# Get all times and coordinates via the respective xpath}
\NormalTok{times <-}\StringTok{ }\KeywordTok{xpathSApply}\NormalTok{(parsed_file, }\DataTypeTok{path =} \StringTok{"//trkpt/time"}\NormalTok{, xmlValue)}
\NormalTok{coords <-}\StringTok{ }\KeywordTok{xpathSApply}\NormalTok{(parsed_file, }\DataTypeTok{path =} \StringTok{"//trkpt"}\NormalTok{, xmlAttrs)}

\CommentTok{# Extract latitude and longitude from the coordinates}
\NormalTok{lats <-}\StringTok{ }\KeywordTok{as.numeric}\NormalTok{(coords[}\StringTok{"lat"}\NormalTok{,])}
\NormalTok{lons <-}\StringTok{ }\KeywordTok{as.numeric}\NormalTok{(coords[}\StringTok{"lon"}\NormalTok{,])}
\end{Highlighting}
\end{Shaded}

\hypertarget{putting-values-into-dataframe}{%
\subsection{Putting values into
dataframe}\label{putting-values-into-dataframe}}

This allows us to have all of the GPX file in one place, ready to be
placed onto a map.

\begin{Shaded}
\begin{Highlighting}[]
\NormalTok{geodf <-}\StringTok{ }\KeywordTok{data.frame}\NormalTok{(}\DataTypeTok{lat =}\NormalTok{ lats, }\DataTypeTok{lon =}\NormalTok{ lons, }\DataTypeTok{time =}\NormalTok{ times)}
\end{Highlighting}
\end{Shaded}

\hypertarget{querying-map-background}{%
\subsection{Querying map background}\label{querying-map-background}}

I used my Google API to access the static map used below.

\begin{Shaded}
\begin{Highlighting}[]
\NormalTok{street <-}\StringTok{ }\KeywordTok{get_map}\NormalTok{(}\DataTypeTok{location =} \StringTok{"409 N 15th St., Lockeland Springs, Nashville, Tennessee"}\NormalTok{,}
                  \DataTypeTok{zoom =} \DecValTok{19}\NormalTok{,}
                  \DataTypeTok{maptype =} \StringTok{"roadmap"}\NormalTok{)}
\end{Highlighting}
\end{Shaded}

\hypertarget{finished-product}{%
\subsection{Finished product}\label{finished-product}}

\begin{Shaded}
\begin{Highlighting}[]
\CommentTok{# Plotting points}
\NormalTok{path <-}\StringTok{ }\KeywordTok{ggmap}\NormalTok{(street) }\OperatorTok{+}
\StringTok{  }\KeywordTok{geom_point}\NormalTok{(}\DataTypeTok{data =}\NormalTok{ geodf,}
             \KeywordTok{aes}\NormalTok{(}\DataTypeTok{x =}\NormalTok{ lon, }\DataTypeTok{y =}\NormalTok{ lat),}
             \DataTypeTok{size =} \DecValTok{1}\NormalTok{,}
             \DataTypeTok{alpha =} \FloatTok{0.7}\NormalTok{,}
             \DataTypeTok{color =} \StringTok{"red"}\NormalTok{)}

\CommentTok{# Adding details}
\NormalTok{path <-}\StringTok{ }\NormalTok{path }\OperatorTok{+}
\StringTok{  }\KeywordTok{labs}\NormalTok{(}\DataTypeTok{x =} \StringTok{"Longitude"}\NormalTok{,}
       \DataTypeTok{y =} \StringTok{"Latitude"}\NormalTok{,}
       \DataTypeTok{title =} \StringTok{"Walking Path Plotted using myTracks"}\NormalTok{,}
       \DataTypeTok{subtitle =} \StringTok{"Carmen Canedo"}\NormalTok{,}
       \DataTypeTok{caption =} \StringTok{"STAT 202, Summer 2020"}\NormalTok{)}

\NormalTok{path}
\end{Highlighting}
\end{Shaded}

\includegraphics{lab1_files/figure-latex/unnamed-chunk-14-1.pdf}

\hypertarget{conclusion}{%
\section{Conclusion}\label{conclusion}}

\begin{itemize}
\tightlist
\item
  What was learned
\item
  Et ecetera
\end{itemize}

\end{document}
